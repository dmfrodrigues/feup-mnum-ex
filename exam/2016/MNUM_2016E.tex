\documentclass{mnum}
\usepackage{hyperref}
\usepackage[portuguese]{babel}
\title{MNUM -- Exam 2016/17}
\author{Diogo Miguel Ferreira Rodrigues (dmfrodrigues2000@gmail.com)}
% Document
\begin{document}

\setcounter{chapter}{15}
\exam{Exame 2016/17}

\question{Pergunta 1}
\begin{enumerate}
    \item
    \textbf{Vantagens}: uma iteração custa uma só soma; evita erros de arredondamento nem de truncatura (dado que operações com inteiros devolvem resultados exatos), apenas possivelmente de \textit{overflow}.\\
    \textbf{Desvantagens}: pouco útil, dado que, na maior parte dos casos úteis em que métodos numéricos são aplicáveis, o valor-base $x$ e o incremento $i$ são floats, que não são representáveis com a precisão decimal necessária.
    \item 
    \textbf{Vantagens}: uma iteração custa uma só soma; maior flexibilidade de representação numérica dado que se tratam de floats.\\
    \textbf{Desvantagens}: operações em floats são mais custosas do que em inteiros; se $x$ for significativamente maior que o incremento $h$, em cada soma o incremento $h$ é truncado para somar com $x$, o que pode introduzir grandes erros depois da soma repetiva do incremento $h$.
    \item
    \textbf{Vantagens}: maior flexibilidade de representação numérica dado que se tratam de floats; Acumula menos erro do que a incrementação sucessiva do item anterior, ao multiplicar o incremento $h$ por um índice, e só depois é que esse valor, significativamente maior do que $h$,é somado a $x$, reduzindo os erros por truncatura por diminuir o número de ordens de grandeza entre os valores que são somados.\\
    \textbf{Desvantagens}: uma iteração custa uma multiplicação e uma soma.
    \item
    \textbf{Vantagens}: pode ter uma vantagem em determinados casos, nos quais o objeto de estudo (geralmente uma função unidimensional neste caso) tem um comportamento particular e interessante quando $x$ é próximo de $x_0+1$ (por exemplo, na integração de uma função contínua com um valor muito maior quando $x$ é aproximadamente $x_0+1$ do que quando comparado com o resto do seu domínio).\\
    \textbf{Desvantagens}: Não ultrapassa o valor $x_0+1$; depois de poucas iterações (cerca de 60 num float de 64 bits), o incremento já não é representável através de floats por ser demasiado pequeno.
\end{enumerate}

\question{Pergunta 2}
\lstinputlisting[language=Maxima, caption=Maxima input 2016E-1]{2016E_2.mc}
\lstinputlisting[language=Python, caption=Programa 2016E-2 (Python3)]{2016E_2.py}
\begin{center}
    \begin{tabular}{c | c}
        $x_n$ & $y_n$ \\ \hline
        1.00000 & 1.00000 \\
        \textbf{2.13333} & \textbf{2.06667} \\
        \textbf{1.74580} & \textbf{1.69764}
    \end{tabular}
\end{center}

\question{Pergunta 3}
\questionitem{Item a}
Método de Levenberg-Marquardt, que consiste em utilizar em simultâneo o passo de quádrica e o passo de gradiente associado a um fator $\lambda$ que diminui quando o passo é bem-sucessido (diminui importância do gradiente), e aumenta quando o passo é mal-sucedido (aumenta importância do gradiente).

\questionitem{Item b}
As principais vantagens são:
\begin{itemize}
    \item Bom desempenho geral, tanto perto como longe do mínimo, dado que associa duas técnicas adequadas a cada uma das situações, atribuindo-lhes pesos diferentes consoante o processo se está a desenrolar e a fase em que se encontra (longe ou perto do mínimo).
    \item Muito melhor desempenho em situações difíceis, o que não se verifica especificamente neste caso dado que não existem zonas alongadas com valores da função próximos do mínimo.
\end{itemize}

\questionitem{Item c}
O principal Pergunta é tratar-se de programação não-convexa (ou seja, a função não possui apenas um mínimo local na região em análise). Uma forma de resolver o Pergunta é realizar uma pesquisa "linear" pelos valores da função, e escolher como ponto de partida o ponto analizado em que o valor da função é menor; ou representar a função graficamente, e delegar a escolha de um guess adequado ao operador da máquina. Neste caso, dado que o enunciado fornece o gráfico da função, o operador pode identificar um guess que, com grande probabilidade e uma escolha correta de parâmetros, conduza ao mínimo absoluto da função.\par
Para lidar com este Pergunta, podemos ainda adotar a perspetiva de que pretendemos encontrar apenas um mínimo local, não necessariamente um mínimo absoluto, e terminarmos a pesquisa ao encontrar um mínimo local qualquer.

\newpage

\question{Pergunta 4}
\lstinputlisting[language=Python, caption=Programa 2016E-4 (Python3)]{2016E_4.py}
\begin{center}
    \begin{tabular}{c | c | c | c | c | c}
        $x_e$ & $x_d$ & $x_n$ & $f(x_e)$ & $f(x_d)$ & $f(x_n)$ \\ \hline
        0.00000 & 0.80000 & 0.65604 & -0.50000 & 0.10972 & -0.11967 \\
        \textbf{0.65604} & \textbf{0.80000} & \textbf{0.73115} \\
        \textbf{0.73115} & \textbf{0.80000} & \textbf{0.74296} \\
        \textbf{0.74296} & \textbf{0.80000} & \textbf{0.74476} \\
    \end{tabular}
\end{center}
\begin{center}
    \begin{tabular}{l | l}
        Critério de paragem & Aplicabilidade \\ \hline
        $|f(x_n)          | \leq \varepsilon_{max}$ & Fundamental \\
        $|f(x_n)-f(x_{n-1}| \leq \varepsilon_{max}$ & Não se aplica \\
        $|x_n-x_{n-1}     | \leq \varepsilon_{max}$ & Aplica-se \\
        $|x_n-x_d| \leq \varepsilon_{max}$ ou $|x_n-x_e| \leq \varepsilon_{max}$ & Aplica-se
    \end{tabular}
\end{center}

\question{Pergunta 5}
\begin{alignat*}{2}
    \frac{d^2y}{dt^2}=A+t^2+t\frac{dy}{dt} \iff
    \begin{cases}
        \frac{dy}{dt} = v \\
        \frac{dv}{dt} = A+t^2+tv
    \end{cases}
\end{alignat*}
\lstinputlisting[language=Python, caption=Programa 2016E-5 (Python3)]{2016E_5.py}
\begin{center}
    \begin{minipage}[c]{0.5\textwidth}
        \begin{center}
            Método de Euler:\\
            \begin{tabular}{c | c c}
                $n$ & $t$ & $y$ \\ \hline
                0 &      \textbf{0.00000} & \textbf{0.00000} \\
                1 &      \textbf{0.25000} & \textbf{0.25000} \\
                2 &      \textbf{0.50000} & \textbf{0.53125}
            \end{tabular}
        \end{center}
    \end{minipage}%
    \begin{minipage}[c]{0.5\textwidth}
        \begin{center}
            Método de Runge-Kutta de 4ª ordem:\\
            \begin{tabular}{c | c c}
                $n$ & $t$ & $y$ \\ \hline
                0 &      \textbf{0.00000} & \textbf{0.00000} \\
                1 &      \textbf{0.25000} & \textbf{0.26874} \\
                2 &      \textbf{0.50000} & \textbf{0.59220}
            \end{tabular}
        \end{center}
    \end{minipage}
\end{center}

\question{Pergunta 6}
\lstinputlisting[language=Maxima, caption=Maxima input 2016E-6]{2016E_6.mc}
\lstinputlisting[language=Python, caption=Programa 2016E-6 (Python3)]{2016E_6.py}
\begin{center}
    \begin{tabular}{l | c | c}
                                       & M. Trapézios & M. Simpson \\ \hline
        $h  $                          & 0.50000     &    0.50000 \\
        $h' $                          & \textbf{0.25000}     &    \textbf{0.25000} \\
        $h''$                          & 0.12500     &    0.12500 \\
        Comprimento do arco $L_1=l$    & \textbf{20.18313}    &    \textbf{19.32073} \\
        Comprimento do arco $L_2=l'$   & \textbf{19.51436}    &    \textbf{19.29144} \\
        Comprimento do arco $L_3=l''$  & \textbf{19.34573}    &    \textbf{19.28952} \\
        Quociente de convergência $QC$ & \textbf{3.96578}     &    \textbf{15.22515} \\
        Erro estimado $\varepsilon$    & \textbf{-0.05621}    &    \textbf{-0.00013}
    \end{tabular}
\end{center}

\end{document}
