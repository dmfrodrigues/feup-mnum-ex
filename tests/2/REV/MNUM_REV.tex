\documentclass{mnum}
\usepackage[portuguese]{babel}
% Metadata
\title{MNUM Teste 2 -- Revisões}
\author{Diogo Miguel Ferreira Rodrigues \\ \href{mailto:up201806429@fe.up.pt}{up201806429@fe.up.pt}}
% Document
\begin{document}

\setcounter{chapter}{0}
\exam{Teste 2 Revisões}

\question{Pergunta 1}
%\lstinputlisting[language=Python, caption=Programa REV-1 (Python3)]{REV_1.py}
\begin{equation*}
	x_0 = \begin{bmatrix}
		4.00000 \\
		8.00000 \\
		-2.00000
	\end{bmatrix}
\end{equation*}
\questionitem{Item a}
A estabilidade interna para os erros apresentados no enunciado é
\begin{equation*}
	\delta x = \begin{bmatrix}
		0.44155 \\
		0.61528 \\
		-0.10858
	\end{bmatrix}
\end{equation*}
Através desta breve análise, pode-se concluir que $x_3$ é a incógnita menos afetada por erros externos, seguida de $x_1$ e de $x_2$ por esta ordem.
\questionitem{Item b}
Os resíduos são
\begin{equation*}
	\varepsilon _0 = \begin{bmatrix}
		0.00000e+00 \\
		-7.10543e-15 \\
		0.00000e+00
	\end{bmatrix}
\end{equation*}
A estabilidade interna é
\begin{equation*}
	\delta = \begin{bmatrix}
		2.47642e-16 \\
		2.28593e-16 \\
		-8.76273e-16
	\end{bmatrix}
\end{equation*}
Estes valores da estabilidade interna indiciam que $x_3$ é a incógnita mais afetada por erros de arredondamento associados ao algoritmo, seguida de $x_1$ e de $x_2$.

\end{document}
